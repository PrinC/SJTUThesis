\chapter{相关工作}
\label{chap:related-work}

本章将对相关工作进行介绍,在AppShield牵涉的领域,已有若干学者有一定的研究与成果,本章主要阐述已有的相关工作以及其优缺点。
~\ref{sec:related:privacy-protect}节、~\ref{sec:related:clone-app}节、~\ref{sec:related:third-lib}节分别介绍了Android隐私保护、克隆应用检测以及第三方库检测的相关工作,~\ref{sec:related:conclusion}节对本章进行小结。

\section{Android隐私数据保护}
\label{sec:related:privacy-protect}

近年来,除开AppAudit之外,也有很多研究将程序分析的方法引入Android隐私数据保护领域中。

AppIntent~\supercite{appintent}使用的是符号执行的方法对应用进行静态程序分析,它对传统的符号执行进行了优化,通过建立Android生命周期的模型,有效的减少了代码的执行路径。
即便如此,AppIntent在16核的机器上分析一个应用的时间仍然长达数十分钟,有的甚至超过了一个小时。
AppIntent另外一个缺陷就是内存占用过高,在分析中,AppIntent经常会占用内存高达32GB,这在个人计算机上是很难应用的。

FlowDroid~\supercite{flowdroid}利用了Soot注入框架~\supercite{soot}实现了对Android应用的静态污点分析。
FlowDroid通过使用Dexpler将应用的dex代码转换成Jimple三地址码。
然后通过建立函数调用图,最后进行污点分析。

TaintDroid~\supercite{taintdroid}通过修改Android中的Dalvik虚拟机,实现了动态的污点分析功能,能让用户在运行时追踪手机敏感信息的走向。
但是TaintDroid会在运行是带来14\%的开销,这对用户使用手机影响较大。

\section{克隆应用检测}
\label{sec:related:clone-app}

DroidMoss~\supercite{droidmoss}利用模糊哈希(Fuzzy Hashing)的方式来检测重打包应用。
模糊哈希是一种能应对内容增减的哈希方法。
它首先将字节码进行稳定化,去除易变的寄存器,只保留相对稳定的操作数(opcode);
另外它将哈希内容分段,同时利用滚动哈希的方法,使得在哈希内容发生增减时,只有增减前后的哈希段会有差异,因此能使用相似哈希段占总共哈希段数的比例来判断应用是否为重打包应用。
此方法的好处在于对于一个应用只需要一次模糊哈希,比对时只需要比对哈希结果,另外对于两个有序列表的共同元素查找的复杂度不高,可以接受。
缺点是此方法依赖于字节码的稳定性,虽然模糊哈希已经对字节码进行了处理,但是字节码操作数在混淆中发生变化的情况不是非常罕见。
所以在针对混淆应用的检测时,漏报率可能比较高。
另外此方法是基于两两比较,所以当需要处理海量应用时,扩展性较差。

JuxtApp~\supercite{juxtapp}使用类似的字节码处理方法,保留操作数,对处理后的字节码使用k-gram和基本块进行分段,并对分段后的字节码进行特征哈希,最后通过对比应用的特征向量来判断应用是否有相似部分。
不同的是,JuxtApp主要针对的是识别存在问题代码的应用。

WuKong~\supercite{wukong}通过两阶段检测来识别克隆应用。
WuKong在第一阶段会简单计算出不同Android API调用的次数,将提取出的向量进行编辑距离的计算,只有低于阈值的应用对才会进行下一步的检测。
由于市场上大多数应用都不是克隆应用,所以在这一步可以剔除大量的非克隆应用。
在第二步中,WuKong将对应用进行基于Boreas~\supercite{boreas}的代码克隆检测方法。
在这一步中,将详细的计算代码的特征并进行比较,最终得出潜在的克隆应用。
WuKong对简单的标识符重命名的混淆应用是有效的,但是由于WuKong会依赖于具体的代码(例如Android API调用),所以当代码发生变化时,比如代码发生缩减或者由于代码优化,Android API调用被剔除,检测会失效。

\section{第三方库检测}
\label{sec:related:third-lib}

在克隆应用检测的研究中,通常需要剔除掉第三方库来增加检测的准确性,之前的很多研究~\supercite{grace2012unsafe,book2013longitudinal,chen2014achieving,stevens2012investigating,shekhar2012adsplit}都是通过设置白名单的方式来进行第三方库的检测。
研究者通过经验或者数据总结出使用较多的第三方库,把这些第三方库的包名记录进白名单。
在进行程序分析的时候,来比对代码中是否存在对应白名单中的包名,把出现在白名单中的包名排除分析内容,从而增加准确性。
这种直接的方式好处在于效率高,并且实现简单;
但是这种做法需要长期维护第三方库的白名单,并且由于白名单的比对依赖于包名,所以当应用发生混淆,则白名单就会失效。
所以利用白名单进行第三方库检测在第三方库迭代更新快并且在混淆环境下会失效。

LibRadar~\supercite{libradar}利用聚类的方法,对应用进行第三方库的检测。
LibRadar会计算出应用不同Android API调用的次数,将其提取出特征向量。
然后使用此特征向量进行聚类分析,最终把分析得出的集群识别为第三方库。
由于LibRadar是使用Android API当做特征,所以只能应对标识符重命名的混淆技术,当代码缩减或者由于代码优化,Android API调用被剔除时,检测会失效。

LibScout~\supercite{libscout}利用对应用代码进行指纹生成与比对,来对应用进行第三方库检测。
但由于LibScout在指纹比对时是以类为粒度,并且在比对时使用了包结构的信息,从而当应用因为代码缩减或者代码优化导致应用的方法被移除和当应用使用了包结构重组织的混淆方法时,检测会失效。

\section{本章小结}
\label{sec:related:conclusion}

在本章中,我们对Android隐私保护、克隆应用检测以及第三方库检测的相关研究工作分别进行了阐述,并总结了不同研究各自的优缺点。
对于Android隐私保护,目前的研究除了AppAudit之外,其余的方法在准确性、分析速度、内存占用等方面不能满足大规模应用审查的应用场景;
对于克隆应用与第三方库检测,相关研究工作都无法应对混淆或者只能应对简单的标识符重命名的混淆技术,其他混淆技术如包结构重组织、代码缩减等均会令检测失效。