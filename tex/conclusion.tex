\chapter{总结与展望}
\label{chap:conclusion}

本章对全文进行总结与展望,主要包括对本文所做贡献进行总结,分析系统中现有的不足之处,对可改进之处进行展望。

\section{总结}

随着移动互联网的爆炸式发展,移动应用的使用越来越频繁,而移动应用的数量也在成几何增长。
然而移动应用的质量参差不齐,其中也不乏有很多盗取用户隐私恶意应用和克隆应用。
为了解决这个问题,我们设计了AppShield大规模应用市场审查系统,并实现了原型。
AppShield使用了兼顾准确率和分析时间的单应用审查工具AppAudit,对单应用进行审查。
之后AppShield会根据单应用审查结果进行跨应用的分析,找出问题部件,并对问题部件进行更深一步的分析。
另外AppShield提出了一种相似度检测的方法,能够在混淆的情况下,对应用进行相似度检测,AppShield中第三方库和克隆应用的检测就是基于此方法。
相似度检测主要分为两阶段,指纹提取和指纹比对,该方法可以应对大多数混淆技术,如标识符重命名、代码缩减、代码优化等等。

在测试评估中,我们分别对AppShield两年来采集的13万多的应用进行了隐私问题检测,
最终找出了1200多例存在隐私问题的应用,并且总结出超过20个有隐私问题的问题部件。
并且我们对测试集中来自腾讯应用宝的10993个应用进行了第三方库的检测,实验表明我们的相似度检测方法能够较好的应用与第三方库的检测中。
另外我们还随机挑选了1000个应用进行了克隆应用的检测,实验找出了一例开发者多次以不同应用名提交应用的事例。
实验数据表明我们的克隆应用检测算法误报率偏高,还有待改善。

\section{展望}

虽然本文已经设计并实现了一个大规模应用审查系统的原型,但还是存在很多可以改进的地方。
\begin{itemize}
	\item 隐私泄露总览以及问题部件总结不够自动化,目前隐私泄露总览需要定期人工使用脚本进行统计,而问题部件的分析需要人工对代码进行分析。
	虽然在大规模应用审查系统中,这些操作不是特别频繁,但是如果能够做到实时图表可视化,以及定期对重点问题部件的问题代码片段进行自动的提取并分析,可以使得应用市场更加自动化,并且能够实时监控应用市场的隐私泄露状况。
	\item 目前相似度分析还无法处理重打包类的混淆技术,但是本文观察到了一些使用重打包类混淆技术的特点,可以基于这些特点,发明启发式算法对重打包类进行识别。
	\item 我们相似度分析是基于两两比较进行的,在应用与第三方库检测时,还能应对大规模的分析。
	当进行克隆应用检测时,此方法的效率在应用规模达到一定程度时就会失效。
	要解决这个问题,可以提高检测过程的并行度,并且将包、类的比较结果缓存下来,使用包和类的哈希进行索引。
	在比对中,如果发现比对双方已有比对结果,就直接使用缓存。
	由于在市场上,应用普遍使用第三方库,所以此优化可以大大降低指纹比对的时间。
	\item 对于第三方库的检测,本文只是做了简单样本的实验。
	在今后的工作中,可以建立起第三方库的数据库,提高第三方库的样本数量,从而使得第三方库的检测更加完备,为应用市场提供更多的有效信息。
	\item 目前在克隆应用检测的实验中,发现误报率很高。
	主要是因为在进行相似度检测时没有剔除第三方库,导致使用相同第三方库的应用相似度会偏高。
	在建立了相对完备的第三方库之后,可以先对应用进行第三方库的检测,从而剔除掉应用包含的第三方库,保留应用本身的逻辑,从而降低误报率。
\end{itemize}

